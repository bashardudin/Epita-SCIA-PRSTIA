\documentclass[11pt, a4paper]{article}

\usepackage[french]{babel}
\usepackage{fancyhdr}
\usepackage[margin=.8in]{geometry}

\usepackage{Style/TeXingStyle}

\pagestyle{fancy}
\renewcommand{\headrulewidth}{1.5pt}
\renewcommand{\footrulewidth}{0.5pt}
\fancyhead[L]{EPITA\_ING2\_2019\_S8}
\fancyhead[R]{Majeure SCIA}
\fancyhead[C]{PRST}
\fancyfoot[C]{\thepage}
\fancyfoot[L]{2018}
\fancyfoot[R]{\textbf{Chargé de cours :} \textsc{Bashar~DUDIN}}

\pretitle{\vspace{-2.5\baselineskip} \begin{center}}
\title{%
  { \huge Feuille d'exercices}%
}
\posttitle{
\end{center}
\rule{\textwidth}{1.5pt}
\vspace{-3\baselineskip}
}
\author{}
\date{}

\pdfinfo{
   /Author (Bashar Dudin)
   /Title  (Exercices PRST - 2018)
   /Subject (Probabilités)
}

\begin{document}

\maketitle\thispagestyle{fancy}

\section{S'habituer au formalisme}
\label{sec:formalisme}


\subsection{Lois à densités -- Manipulation}

\noindent On a aborde quelques exercices simples de manipulation des lois à
densités. Ces lois nécessitent une certaine aisance dans l'utilisation
des intégrales généralisées.

\paragraph{Deux petits exemples.}

\begin{question}
  On considère une variable aléatoire $X$ suivant la loi uniforme sur
  $[0, 1]$. Quelle est la loi de la variable aléaoitre $Y = X(X-1)$?
\end{question}

\begin{solution}
  L'expressions $x(x-1)$ décrit une parabole dont le sommet est
  atteint en $1/2$ et y prend la valeur $-1/4$. Sur $[0, 1]$ cette
  parabole prend donc toutes les valeurs dans $[-1/4, 0]$. Comme $X$
  prend ses valeurs dans $[0, 1]$, la V.A. Y prend ses valeurs dans
  $[-1/4, 0]$. On va décrire la fonction de répartition de $Y$. Soit
  $t \in [-1/4, 0]$.
  \[
  Y \leq t \Longleftrightarrow X^2 - X - t \leq 0.
  \]
  Pour décrire le lieu $Y \leq t$ on étudie donc le signe du trinôme
  du seconde degré $X^2 - X - t$ en $X$. Ce trinôme est négatif entre
  les racines. Une étude du discriminanet permet de voie que ces
  racines sont
  \[
  \frac{1 - \sqrt{1+4t}}{2} \quad \textrm{et} \quad \frac{1 +
    \sqrt{1+4t}}{2}.
  \]
  Ainsi
  \[
  Y \leq t \Longleftrightarrow \frac{1 - \sqrt{1+4t}}{2} \leq X \leq
  \frac{1 + \sqrt{1+4t}}{2}.
  \]
  Donc
  \[
  \mbb{P}(Y \leq t) = \mbb{P}\left(\frac{1 - \sqrt{1+4t}}{2} \leq X \leq \frac{1 +
    \sqrt{1+4t}}{2}\right) = \sqrt{1+4t}.
  \]
  La densité $f_Y$ de $Y$ est ainsi donnée par
  \[
  f_y(x) = \frac{2}{\sqrt{1 + 4x}}\bOne_{[-1/4, 0]}(x).
  \]
  Pour le voir il suffit de voir le fait que la densité est la dérivée
  de la fonction de répartition là où celle-ci est dérivable. Comme la
  valeur n'a pas d'importance dans le seul point où elle ne l'est pas
  (en $0$) on obtient ce qu'on cherche.
\end{solution}

\begin{question}
  On considère une expérience aléatoire où l'on choisit au hasard un
  point du cercle de centre l'origine et de rayon l'unité. On note $X$
  et $Y$ les variables aléatoires réelles qui correspondent au choix
  des coordonnées d'un tel point. Décrire les lois de $X$ et $Y$.

  \noindent Pourriez-vous généraliser au cas d'un point de la sphère?
\end{question}

\begin{solution}
  On repère le point du cercle unité par l'angle que forme le rayon
  lui correspondant avec l'axe des abscisses. Autrement dit par ses
  coordonnées polaires. L'angle $\theta$ qui repère un point du cercle
  unité varie dans $[0, 2\pi[$. On peut donc décrire une probabilité
  sur le cercle unité par l'espace d'états $\Omega = [0, 2\pi[$, muni
  de la tribu des boréliens, et agrémenté de la probab $(1/2\pi)\ell$
  où $\ell$ est déterminée sur les intervalles de $\Omega$ par la
  longueur de ceux-ci.

  \noindent La variable aléatoire $X$ est décrite par ce cadre par la
  fonction $\theta \mapsto \cos(\theta)$ et qui prend ses valeurs dans
  $[-1, 1]$. La fonction de répartition de $X$ est décrite pour tout
  $t \in [-1, 1]$ par
  \[
  \mbb{P}(X \leq t) = \ell\left(\big[\theta \mid \cos(\theta) \leq
    t\big]\right).
  \]
  On pourrait s'arrêter à ce stade, car vous avez jamais vu la
  fonction $\arccos$. On va, cela dit, aller un peu plus loin. La
  fonction $\arccos$ est la réciproque de la fonction $\cos$ quand
  cette dernière est restreinte à l'espace de départ $[0, \pi]$. Sur
  cet interval $\arccos$ est décroissante. On a donc, pour tout
  $t \in [-1, 1]$,
  \[
  \theta \in [0, \pi] \, \textrm{et} \, \cos(\theta) \leq t
  \Longleftrightarrow \arccos(t) \leq \theta \leq \pi.
  \]
  La symétrie axiale du graphe de $\cos$ par rapport à l'axe
  $x = \pi$\footnote{Il faut un peu de travail et de dessin pour le
    voir. Je compte sur vous. (Oui, j'ai la flemme, mais venez me
    poser des questions).} permet de voir que
  \[
  \theta \in [0, 2\pi] \, \textrm{et} \, \cos(\theta) \leq t
  \Longleftrightarrow \arccos(t) \leq \theta \leq 2\pi - \arccos(t).
  \]
  D'où, pour tout $t \in [-1, 1]$,
  \begin{align*}
    \mbb{P}(X \leq t) & = \mbb{P}\left(\arccos(t) \leq \theta \leq 2\pi - \arccos(t)\right) \\
                      &  = 1 - \frac{1}{\pi}\arccos(t).
  \end{align*}
  Hors de $[-1, 1]$ la fonction de répartition de $X$ est, ou bien
  nulle quand $t < -1$ ou égale à $1$ sinon. Cela suffit pour
  déterminer la loi de $X$ ainsi que celle de $Y$ (par symétrie, c'est
  la même que la loi de $X$).

  \noindent On peut ajouter à l'étude le calcul de la densité de la
  fonction de répartition $F_X$. C'est une fonction dérivable sauf
  éventuellement en $-1$ et $1$. En dehors de ces points on obtient
  pour fonction de densité
  \[
  f_X(x) = \frac{1}{\pi\sqrt{1-x^2}}\bOne_{[-1, 1]}(x).
  \]
  qu'on prolonge par $0$ en $-1$ et $1$. Cette relation vient du fait que
  \[
  \arccos'(x) = \frac{1}{\sqrt{1-x^2}}\footnote{Si vous voulez savoir
    pourquoi posez-moi la question.}.
  \]
\end{solution}


\paragraph{Loi exponentielle et temps d'arrêt.}

La date de connexion d'un client à votre après $00:00$ est une
variable aléatoire définie sur un espace probabilisé
$(\Omega, \mc{A}, \mbb{P})$, de loi exponentielle de paramètre
$\lambda$.
\begin{question}
  \begin{enumerate}
  \item
    Rappeler et calculer les moments d'ordre $1$ et $2$ de d'une loi
    exponentielle de paramètre $\lambda$. En déduire la variance d'une
    telle loi.
  \item
    Calculer la probabilité $\mbb{P}(T > 1/\lambda)$.
  \item
    On fixe $\epsilon > 0$. On considère pour $k \in \N$ les plages
    horaires $I_k = [k\epsilon, (k+1)\epsilon[$. Calculer la
    probabilité $\mbb{P}(T \in I_k)$.
  \item
    Soit $X$ la variable aléatoire à valeurs dans $\N$ définie pour
    tout $\omega in \Omega$ par
    \[
    X(\omega) = k \Leftrightarrow T(\omega) \in I_k.
    \]
    Quelle est la loi de $X$?
  \item
    Calculer pour tout $t > 0$ et tout $h > 0$, la probabilité
    $\mbb{P}(t < T)$ ainsi que la probabilité conditionnelle
    $\mbb{P}(T > t+h \mid T > t)$. Qu'est-ce que cela signifie?
  \end{enumerate}
\end{question}

\paragraph{Somme de densités.} On considère deux variables aléatoires
$X$ définies sur $\Omega$, on construit à partir de celles-ci la
variable aléatoire sur $\Omega \times \{0, 1\}$ définie par:
\[
X(\omega, 0) = X_1(\omega) \quad\textrm{et}\quad X(\omega, 1) = X_2(\omega)
\]
\begin{question}
  Exprimer la loi de probabilités de $X$ en fonction de $X_1$ et
  $X_2$. Si $X_1$ et $X_2$ sont des lois à densité montrer qu'il en va
  de même de $X$ et calculer sa densité.
\end{question}

\paragraph{Transfert.} On considère la variable aléatoire $X$ à valeurs
dans $]0, 1[$ et dont la densité est donnée pour $x \in \R$ par
\[
f_X(x) = \frac{1}{\ln(2)}\times \frac{1}{1+x}\bOne_{]0, 1[}(x).
\]
\begin{question}
  \begin{enumerate}
  \item
    Déterminer la loi de $Y = 1/X$.
  \item
    On note $\mbb{E}(Y)$ la partie entière de $Y$, déterminer la loi de
    $Z = Y - \mbb{E}(Y)$.
  \end{enumerate}
\end{question}

\subsection{Lois à densités -- Moments}

\paragraph{Loi de Pascal.}
On considère la variable aléatoire $X$ suivant la loi de densité
$f_X(x) =  \frac{1}{2}e^{-|x|}$.
\begin{question}
  Calculer l'espérence de la variable aléatoire
  $Y = \bOne_{[1, +\infty[}X + \bOne_{[-\infty, 1]}$.
\end{question}

\begin{solution}
  D'après le théorème de transfert on a
  \[
  \mbb{E}(Y) = \int_\R \Big(x\bOne_{[1, +\infty[}(x) + \bOne_{[-\infty, 1]}(x)\Big)\frac{1}{2}e^{-|x|}\dd{x}.
  \]
  D'où
  \[
  \mbb{E}(Y) = \frac{1}{2}\int_1^{+\infty}xe^{-x}\dd{x} + \frac{1}{2} \int_{-\infty}^0 e^{x}\dd{x} + \frac{1}{2}\int_{0}^1 e^{-x}\dd{x}.
  \]
  Une intégration par partie pour la première intégrale et un calcul
  direct pour la seconde donne
  \[
   \mbb{E}(Y) = 1 - \frac{1}{e}.
  \]
\end{solution}

\paragraph{Loi de Rayleigh}
On considère la variable aléatoire $X$ suivant la loi de densité
$f_X(x) = xe^{-x^2/2}\bOne_{[0, +\infty[}$.
\begin{question}
  \begin{enumerate}
  \item Déterminer l'espérence $\mbb{E}(X)$ de $X$. Calculer
    $\mbb{P}\big(X > \mbb{E}(X)\big)$.
  \item Calculer $\mbb{E}(X^2)$. En déduire $\mbb{V}(X)$.
  \item Calculer $\mbb{R}\big(|X-\mbb{E}(X)| > 1\big)$. Comparer le
    résultat avec l'inégalité de Tchebychev.
  \end{enumerate}
\end{question}

\begin{solution}
  L'espérence de la V.A. $X$ est donnée par l'expression
  \[
  \mbb{E}(X) = \int_0^{+\infty} x^2e^{-x^2/2}\dd{x}.
  \]
  En décomposant l'intégrande en $u = x$ et $v' = xe^{-x^2/2}$ une
  intégration par parties donne
  \[
  \mbb{E}(X) = \sqrt{\frac{\pi}{2}}
  \]
  par comparaison à l'espérence d'une variable aléatoire de loi
  gaussienne ayant une espérence nulle et une variance de $1$.

  \noindent En s'aidant d'une intégration par parties on a
  \[
  \mbb{P}\big(X \geq \mbb{E}(X)\big) =
  \int_{\sqrt{\pi/2}}^{+\infty}xe^{-x^2/2}\dd{x} = e^{-\pi/4}.
  \]
  Une intégration par parties en deux temps qui prolonge le
  raisonnemenet à la première question permet d'avoir
  \[
  \mbb{E}(X^2) = 2
  \]
  donc
  \[
  \mbb{V}(X) = 2 - \frac{\pi}{2}.
  \]
  Essayez de justifier sans calculette le fait que
  $e^{-\pi/4} > \mbb{V}(X)$.

  \noindent Un calcul similaire à ce qui a été effectué jusqu'à
  présent (toujours en complexifiant) donne
  \[
  \mbb{P}\big(|X- \mbb{E}(X)|\big) = 1 +
  e^{-\pi/4}e^{-1/2}\left(e^{-\sqrt{\pi/2}} - e^{\sqrt{\pi/2}}\right).
  \]
  La calculatrice vous permet de voir que c'est un bien meilleur
  encadrement que celui donné par l'inégalité de Tchebychev.
\end{solution}


\subsection{Lois à densités -- Conditionnement}

\paragraph{Manipulation des lois conjointes et marginales.} On
travaille trois exercices dont l'objectif est de vous habituer à la
manipulation des lois conditionnelles et conjointes à denstiés.

\begin{question}
  Soit $(X, Y)$ un couple de variables aléatoires à dans
  $[0, +\infty[$, dont la loi conjointe a pour densité
  \[ f_{X, Y}(x, y) = \frac{1}{2\sqrt{x}}e^{-y}\bOne_D(x, y)
  \]
  où $D$ désigne le lieu de $\R^2$
  $\{(x, y) \mid x > 0, y > \sqrt{x}\}$.
  \begin{enumerate}
  \item Déterminer les lois de $X$ et $Y$.
  \item Les variables $X$ et $Y$ sont elles indépendantes?
  \end{enumerate}
\end{question}

\begin{solution}
  Les lois de $X$ et $Y$ ont des densités à supports dans
  $[0, +\infty]$ (elles sont nulles à l'extérieur de cet
  intervalle). Pour $x \in ]0, +\infty]$, on a:
  \[
  f_X(x) = \int_0^{+\infty} f_{(X, Y)}(x, y)\dd{y} = \int_0^{+\infty}
  \frac{1}{2\sqrt{x}} e^{-y} \bOne_D(x,y)\dd{y} =
  \frac{1}{2\sqrt{x}}\int_{\sqrt{x}}^{+\infty}e^{-y}\dd{y}
  \]
  donc
  \[
  f_X(x) = \frac{1}{2\sqrt{x}}e^{-\sqrt{x}}.
  \]
  Pour $y \in ]0, +\infty]$, on a:
  \[
    f_Y(y)  = \int_0^{+\infty} \frac{1}{2\sqrt{x}}e^{-y}\bOne_D(x, y)\dd{x}
            = \int_0^{y^2} \frac{1}{2\sqrt{x}} e^{-y}\dd{x}
  \]
  et ainsi,
  \[
  f_Y(y)  = ye^{-y}.
  \]

   \noindent Dans la situation où les variables $X$ et $Y$ étaient
   indépendantes, le produit de leurs densités serait égal à la
   densité de la loi conjointe, ce qui n'est pas le cas.
\end{solution}

\begin{question}
  Soient $X$ et $Y$ deux variables aléatoires indépendantes de même
  densité $x \mapsto \frac{1}{\sqrt{2\pi}}e^{-x^2/2}$.
  \begin{enumerate}
  \item Déterminer la loi de la variable aléatoire $R = \sqrt{X^2+Y^2}$.
  \item Déterminer la valeur moyenne de $R$.
  \item Les variables aléatoires $X$ et $R$ sont-elles indépendantes?
  \end{enumerate}
\end{question}

\begin{solution}
  On désigne respectivement par $f_X$, $f_Y$ et $f_{(X, Y)}$ les
  densités de $X$, de $Y$ et de la loi conjointe $(X, Y)$. On cherche
  à calculer la fonction de répartition de $R$. Pour tout
  $t \in \R_+$, la condition
  \[
  R \leq t \Longleftrightarrow X^2 + Y^2 \leq t^2 \Longleftrightarrow
  R \in D(\bs{o}, t)
  \]
  où $D(\bs{o}, t)$ désigne le disque de centre l'origine et de rayon
  $t$. On a ainsi,
  \[
  \mbb{P}(R \leq t) = \int_{D(\bs{o}, t)} f_{(X, Y)}(x, y)\dd{x}\dd{y}
  \]
  Par indépendance la densité de la loi conjointe $(X, Y)$ est le
  produits de celles de $X$ et de $Y$. Donc
  \[
  \mbb{P}(R \leq t) = \int_{D(\bs{o}, t)}
  \frac{1}{\sqrt{2\pi}}e^{-x^2/2}\frac{1}{\sqrt{2\pi}}e^{-y^2/2}\dd{x}\dd{y}
  \]
  Au vu du lieu d'intégration il est naturel de vouloir passer en
  coordonnées polaires. Je rappelle qu'on a, pour pouvoir effectuer ce
  changement de variables, on utilise dans la pratique les relations
  \[
  x = r\cos(\theta), \quad y = r\sin(\theta) \quad \textrm{et} \quad
  \dd{x}\dd{y} = r\dd{r}\dd{\theta}.
  \]
  On en déduit
  \[
  \mbb{P}(R \leq t) = \frac{1}{2\pi}\int_0^{2\pi}\int_0^t
  e^{-r^2/2}r\dd{r}\dd{\theta} = \int_0^te^{-r^2/2}r\dd{r}.
  \]
  La loi de $R$ est donc donné par la densité
  $f_R(r) = re^{-r^2/2}\bOne_{[0, +\infty]}(r)$.

  \noindent L'espérence de $R$ est par définition
  \[
  \mbb{E}(R) = \int_0^{+\infty} e^{-r^2/2}\dd{r}
  \]
  Par symétrie de la densité d'une loi gaussienne d'espérence $0$ et
  de variance $1$, l'espèrence de cette dernière est le double de
  celle qui nous intéresse. Ainsi
  \[
  \mbb{E}(R) = \sqrt{\frac{\pi}{2}}
  \]

  \noindent On ne connaît ici la densité de la loi conjointe de $X$ et
  $R$. Pour montrer que c'est deux lois ne sont pas indépendantes on
  peut procéder en exhibant une évènement de la loi conjointe dont la
  probabilités ne correspond pas au produit des évènements marginaux
  qui le composent. Ici il suffit de voir que
  \[
  \mbb{P}(X > 1, R < 1) = 0
  \]
  mais $\mbb{P}(X > 1) \neq 0$ et $\mbb{P}(R < 1) \neq 0$.
\end{solution}


\begin{question}
  Soit $(X, Y)$ une variable aléatoire à valeurs dans
  $]0, 1[\times ]0, 1[$, de densité
  \[
  f_{(X, Y)}(x, y) = 3\bOne_D
  \]
  où $D$ est le lieu de $\R^2$
  \[
  D = \big\{(x, y) \mid 0 < x^2 < y < \sqrt{x} < 1\big\}.
  \]
  Déterminer les lois conditionnelles et espérences condtionnelles de
  $Y$ par rapport à $X$ et de $X$ par rapport à $Y$.
\end{question}

\begin{solution}
  La densité $f_X$ de $X$ est donnée par
  \[
  f_X(x) = \int_D 3\bOne_D(x, y)\dd{y}.
  \]
  La variable $x$ étant fixée, la fonction caractéristique
  $\bOne_D(x, y)$ vaut $1$ ssi $ y \in [x^2, \sqrt{x}]$ avec
  $x \in [0, 1]$. Donc
  \[
  f_X(x) = \left(\int_{x^2}^{\sqrt{x}} 3\dd{y}\right)\bOne_{]0, 1]}(x)
  \]
  On utilise ici la positivité de l'intégrale (Thm de Fubini) pour
  pouvoir intervertir les éléments d'intégration. On en déduit
  \[
  f_X(x) = 3\left(\sqrt{x} - x^2\right)\bOne_{]0, 1]}(x).
  \]
  Pour le cas de $Y$ il suffit de remarque qu'on peut également écrire
  $D$ sous la forme
  \[
  D = \big\{(x, y) \mid 0 < y^2 < x < \sqrt{y} < 1\big\}.
  \]
  En manipulant l'encadrement dans la définitionde $D$. On en déduit
  donc que
  \[
  f_Y(y) = 3\left(\sqrt{y} - y^2\right)\bOne_{]0, 1]}(y).
  \]
  On en déduit les densités conditionnelles
  \[
  f_{X\mid Y = y}(x) = \frac{f_{(X, Y)}(x, y)}{f_Y(y)} =
  \frac{1}{\sqrt{y} - y^2}\bOne_D(x, y)
  \]
  car l'appartenance à $D$ est plus contraignante que celle donnée par
  $\bOne_{[0, 1]}$. La densité conditionnelle de $Y$ sachant $X = x$
  est donnée par la même expression au noms des variables près. À
  partir de là on trouve
  \[
  \mbb{E}(X \mid Y = y) = \int_\R xf_{X \mid Y = y}(x)\bOne_D{(x,
    y)}\dd{x} = \int_{y^2}^{\sqrt{y}}\frac{x}{\sqrt{y} - y^2}\dd{x}.
  \]
  Un calcul direct donne
  \[
  \mbb{E}(X \mid Y = y) = \frac{\sqrt{y} + y^2}{2}\bOne_{]0,
    1]}(y)\footnote{Rappel: $a^2-b^2 = (a-b)(a+b)$.}.
  \]
  On en déduit
  \[
  \mbb{E}(X \mid Y) = \frac{\sqrt{Y} + Y^2}{2}\bOne_{]0, 1]}(Y).
  \]
  Le calcul dans le cas de $\mbb{E}(Y \mid X)$ est identique.
\end{solution}

\section{De la probabilité en ML}
\label{sec:ML}

\noindent La ML construit des modèles dont l'objectif est un parmi:
\begin{itemize}
\item[\textbullet] prédire une caractéristique d'intérêt d'un individu à partir de
  caractéristiques connues de celui-ci
\item[\textbullet] identifier des \emph{patterns} dans un ensemble de données à
  l'étude, sans que cela soit conduit par un objectif particulier.
\end{itemize}
Les situations qu'on décrit par la suite apparaissent dans le premier
contexte. Dans ce cadre on suppose qu'on a des données qui contiennent
les caractéristiques d'un nombre d'individus ainsi que \emph{la}
caractéristique qu'on souhaite prédire.

\subsection{\emph{Score} d'un classificateur}
\label{sec:classificateurAlea}

On chercher dans cette section à quantifier la \textit{qualité} de
modèles de ML qu'on appelle les classificateur. Un classificateur est
un modèle apparaît dans la situation où la caractéristique qu'on
cherche à prédire est discrète ; par exemple des types de plantes, des
couleurs de cheveux, des appréciations de goûts etc. C'est une
fonction qui étant donné un certaint nombre de caractéristique en
entrée renvoie une valeurs discrète, souvent codées entre $0$ et le
nombre de classes à prédire moins un.

Dans la formalisation qu'on propose ici on considère que les
caractéristique en entrée constitue un espace d'états $\Omega$. On
considère de plus que celui-ci vient avec une fonction $\lambda$ qu'on
désigne par \textit{label} et qui nous indique la caractéristique
qu'on cherche à prédire. On a supposé que le dataset en entrée
contient cette information. Dans ce contexte un classificateur est une
variable aléatoire $X : \Omega \to F$ où $F$ est l'espace des labels
qu'on cherche à attribuer à nos entrées. L'espace $\Omega$ est supposé
fini probabilisé muni d'une probabilité $\mbb{P}$. C'est cette
probabilité qui fait office de \textit{proportion}.

On suppose que $X$ est donnée et on souhaite étudier trois quantités
qui sont liées à l'évaluation de la qualité d'un classificateur.

\paragraph{Cas binaire.} On se limite en un premier temps au cas d'un
classificateur binaire, c'est-à-dire que $F = \{0, 1\}$.
\begin{question}
  On appelle \emph{précision totale} d'un classificateur binaire $X$
  la proportion des individus bien classés, c'est-à-dire ceux où les
  réponses de $X$ et $\lambda$ concident.
  \begin{enumerate}
  \item Exprimer la précision totale de $X$ en s'aidant de la variable
    aléatoire $X - \lambda$.
  \item Déterminer la loi de $X-\lambda$ et interpréter ses deux
    autres valeurs.
  \end{enumerate}
\end{question}
Dans le cas d'un classificateur binaire deux autres quantités
apparaissent relativement naturellement à l'évaluation :
\begin{itemize}
\item la \emph{précision}: proportion des vrais positifs par
  rapport à l'ensemble des bonnes classifications.
\item le \emph{rappel}: proportion des vrais positifs par rapport
  à l'ensemble des positifs (donnés par $\lambda$).
\end{itemize}
\begin{question}
  \begin{enumerate}
  \item Exprmier précision et rappel de $X$.
  \item Étudier l'interconnexion entre précision et rappel ;
    qu'arrive-t-il à l'une si l'autre augmente?
  \end{enumerate}
\end{question}
Dans les problématiques de classification on est toujours attentifs au
fait d'être confrontés à des modèles dont la dépendance au label
$\lambda$ est très faible. C'est pour cette raison qu'on va souvent
chercher à évaluer le \emph{score} d'un classificateur $X$ tels que
$X$ et $\lambda$ définissent des variables aléatoires indépendantes.
\begin{question}
  \begin{enumerate}
  \item Simplifier les expressions de la précision totale, précision
    et rappel dans le cas où $X$ et $\lambda$ sont indépendants.
  \item On suppose que $X$ et $\lambda$ sont indépendants et suivent
    des loi de Bernoulli respectivement de paramètres $p$ et
    $q$. Exprimer les différents scores de $X$ dans ce cas.
  \item Quels resultats numériques obtenez-vous pour chacun des scores
    quand $p=q=0.9$? Qu'en déduisez-vous?
  \end{enumerate}
\end{question}

\paragraph{Cas général.} On revient brièvement vers le cas général. On
suppose désormais que $F$ est l'ensemble discret $\{0, \ldots, k-1\}$.
\begin{question}
  \begin{enumerate}
  \item Exprimer la précision totale de $X$.
  \item Étudiez la loi de $X-\lambda$. Est-elle aussi facilement
    interprétable que dans le cas binaire?
  \item Chercher une variable aléatoire plus adaptée à l'étude des
    problématiques de classifications qui ne sont pas binaires.
  \end{enumerate}
\end{question}


\subsection{Classificateur Bayesien}
\label{sec:Bayesien}

\subsection{\emph{Trade-off} biais variance}
\label{sec:biaisVariance}


\pretitle{\vspace{-2\baselineskip} \begin{center}}
\title{%
  { \huge Solutions des exercices}%
}
\posttitle{
\end{center}
  \vspace{.5\baselineskip}
  \rule{\textwidth}{1.5pt}
  \vspace{-5\baselineskip}
}

\maketitle\thispagestyle{fancy}

\noindent Vous trouverez dans la suite solutions et indications d'une
partie des exercices de la feuille. Ceci étant majoritairement
accessibles il vous est suffisant de comparer votre travail au
résultats que vous retrouverez dans la suite.

\printsolutions

\end{document}

%%% Local Variables:
%%% mode: latex
%%% TeX-master: t
%%% End:
