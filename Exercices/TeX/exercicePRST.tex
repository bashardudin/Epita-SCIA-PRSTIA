\documentclass[11pt, a4paper]{article}

\usepackage[french]{babel}
\usepackage{fancyhdr}
\usepackage[margin=.8in]{geometry}

\usepackage{Style/TeXingStyle}

\pagestyle{fancy}
\renewcommand{\headrulewidth}{1.5pt}
\renewcommand{\footrulewidth}{0.5pt}
\fancyhead[L]{EPITA\_ING2\_2019\_S8}
\fancyhead[R]{Majeure SCIA}
\fancyhead[C]{PRST}
\fancyfoot[C]{\thepage}
\fancyfoot[L]{2018}
\fancyfoot[R]{\textbf{Chargé de cours :} \textsc{Bashar~DUDIN}}

\pretitle{\vspace{-2.5\baselineskip} \begin{center}}
\title{%
  { \huge Feuille d'exercices}%
}
\posttitle{
\end{center}
\rule{\textwidth}{1.5pt}
\vspace{-3\baselineskip}
}
\author{}
\date{}

\pdfinfo{
   /Author (Bashar Dudin)
   /Title  (Exercices PRST - 2018)
   /Subject (Probabilités)
}

\begin{document}

\maketitle\thispagestyle{fancy}

\section{S'habituer au formalisme}
\label{sec:formalisme}


\subsection{Lois à densités -- Manipulation}

\noindent On a aborde quelques exercices simples de manipulation des lois à
densités. Ces lois nécessitent une certaine aisance dans l'utilisation
des intégrales généralisées.

\paragraph{Deux petits exemples.}

\begin{question}
  On considère une variable aléatoire $X$ suivant la loi uniforme sur
  $[0, 1]$. Quelle est la loi de la variable aléaoitre $Y = X(X-1)$?
\end{question}

\begin{question}
  On considère une expérience aléaatoire où l'on choisit au hasard un
  point du cercle de centre l'origine et de rayon l'unité. On note $X$
  et $Y$ les variables aléatoires réelles qui correspondent au choix
  des coordonnées d'un tel point. Décrire les lois de $X$ et $Y$.

  Pourriez-vous généraliser au cas d'un point de la sphère?
\end{question}

\paragraph{Loi exponentielle et temps d'arrêt.}

La date de connexion d'un client à votre après $00:00$ est une
variable aléatoire définie sur un espace probabilisé
$(\Omega, \mc{A}, \mbb{P})$, de loi exponentielle de paramètre
$\lambda$.
\begin{question}
  \begin{enumerate}
  \item
    Rappeler et calculer les moments d'ordre $1$ et $2$ de d'une loi
    exponentielle de paramètre $\lambda$. En déduire la variance d'une
    telle loi.
  \item
    Calculer la probabilité $\mbb{P}(T > 1/\lambda)$.
  \item
    On fixe $\epsilon > 0$. On considère pour $k \in \N$ les plages
    horaires $I_k = [k\epsilon, (k+1)\epsilon[$. Calculer la
    probabilité $\mbb{P}(T \in I_k)$.
  \item
    Soit $X$ la variable aléatoire à valeurs dans $\N$ définie pour
    tout $\omega in \Omega$ par
    \[
    X(\omega) = k \Leftrightarrow T(\omega) \in I_k.
    \]
    Quelle est la loi de $X$?
  \item
    Calculer pour tout $t > 0$ et tout $h > 0$, la probabilité
    $\mbb{P}(t < T)$ ainsi que la probabilité conditionnelle
    $\mbb{P}(T > t+h \mid T > t)$. Qu'est-ce que cela signifie?
  \end{enumerate}
\end{question}

\paragraph{Somme de densités.} On considère deux variables aléatoires
$X$ définies sur $\Omega$, on construit à partir de celles-ci la
variable aléatoire sur $\Omega \times \{0, 1\}$ définie par:
\[
X(\omega, 0) = X_1(\omega) \quad\textrm{et}\quad X(\omega, 1) = X_2(\omega)
\]
\begin{question}
  Exprimer la loi de probabilités de $X$ en fonction de $X_1$ et
  $X_2$. Si $X_1$ et $X_2$ sont des lois à densité montrer qu'il en va
  de même de $X$ et calculer sa densité.
\end{question}

\paragraph{Transfert.} On considère la variable aléatoire $X$ à valeurs
dans $]0, 1[$ et dont la densité est donnée pour $x \in \R$ par
\[
f_X(x) = \frac{1}{\ln(2)}\times \frac{1}{1+x}\chi_{]0, 1[}(x).
\]
\begin{question}
  \begin{enumerate}
  \item
    Déterminer la loi de $Y = 1/X$.
  \item
    On note $\mbb{E}(Y)$ la partie entière de $Y$, déterminer la loi de
    $Z = Y - \mbb{E}(Y)$.
  \end{enumerate}
\end{question}

\subsection{Lois à densités -- Moments}

\paragraph{Loi de Pascal.}
On considère la variable aléatoire $X$ suivant la loi de densité
$f_X(x) =  \frac{1}{2}e^{-|x|}$.
\begin{question}
  Calculer l'espérence de la variable aléatoire
  $Y = \chi_{[1, +\infty[}X + \chi_{[-\infty, 1]}$.
\end{question}

\paragraph{Loi de Rayleigh}
On considère la variable aléatoire $X$ suivant la loi de densité
$f_X(x) = xe^{-x^2/2}\chi_{[0, +\infty[}$.
\begin{question}
  \begin{enumerate}
  \item Déterminer l'espérence $\mbb{E}(X)$ de $X$. Calculer
    $\mbb{P}\big(X > \mbb{E}(X)\big)$.
  \item Calculer $\mbb{E}(X^2)$. En déduire $\mbb{V}(X)$.
  \item Calculer $\mbb{R}\big(|X-\mbb{E}(X)| > 1\big)$. Comparer le
    résultat avec l'inégalité de Tchebychev.
  \end{enumerate}
\end{question}

\subsection{Lois à densités -- Conditionnement}

\paragraph{Manipulation des lois conjointes et marginales.} On
travaille trois exercices dont l'objectif est de vous habituer à la
manipulation des lois conditionnelles et conjointes à denstiés.

\begin{question}
  Soit $(X, Y)$ un couple de variables aléatoires à dans
  $[0, +\infty[$, dont la loi conjointe a pour densité
  \[ f_{X, Y}(x, y) = \frac{1}{2\sqrt{x}}e^{-y}\chi_D(x, y)
  \]
  où $D$ désigne le lieu de $\R^2$
  $\{(x, y) \mid x > 0, y > \sqrt{x}\}$.
  \begin{enumerate}
  \item Déterminer les lois de $X$ et $Y$.
  \item Les variables $X$ et $Y$ sont elles indépendantes?
  \end{enumerate}
\end{question}

\begin{question}
  Soient $X$ et $Y$ deux variables aléatoires indépendantes de même
  densité $x \mapsto \frac{1}{\sqrt{2\pi}}e^{-x^2/2}$.
  \begin{enumerate}
  \item Déterminer la loi de la variable aléatoire $R = \sqrt{X^2+Y^2}$.
  \item Déterminer la valeur moyenne de $R$.
  \item Les variables aléatoires $X$ et $R$ sont-elles indépendantes?
  \end{enumerate}
\end{question}

\begin{question}
  Soit $(X, Y)$ une variable aléatoire à dans $]0, 1[\times ]0, 1[$,
  de densité $f_{(X, Y)}(x, y) = 3\chi_D$ où $D$ est le lieu de $\R^2$
  \[
  D = \big\{(x, y) \mid 0 < x^2 < y < \sqrt{x} < 1\big\}.
  \]
  Déterminer les lois conditionnelles et espérences condtionnelles de
  $Y$ par rapport à $X$ et de $X$ par rapport à $Y$.
\end{question}

\section{De la probabilité en ML}
\label{sec:ML}

\noindent La ML construit des modèles dont l'objectif est un parmi:
\begin{itemize}
\item[\textbullet] prédire une caractéristique d'intérêt d'un individu à partir de
  caractéristiques connues de celui-ci
\item[\textbullet] identifier des \emph{patterns} dans un ensemble de données à
  l'étude, sans que cela soit conduit par un objectif particulier.
\end{itemize}
Les situations qu'on décrit par la suite apparaissent dans le premier
contexte. Dans ce cadre on suppose qu'on a des données qui contiennent
les caractéristiques d'un nombre d'individus ainsi que \emph{la}
caractéristique qu'on souhaite prédire.

\subsection{\emph{Score} d'un classificateur}
\label{sec:classificateurAlea}

On chercher dans cette section à quantifier la \textit{qualité} de
modèles de ML qu'on appelle les classificateur. Un classificateur est
un modèle apparaît dans la situation où la caractéristique qu'on
cherche à prédire est discrète ; par exemple des types de plantes, des
couleurs de cheveux, des appréciations de goûts etc. C'est une
fonction qui étant donné un certaint nombre de caractéristique en
entrée renvoie une valeurs discrète, souvent codées entre $0$ et le
nombre de classes à prédire moins un.

Dans la formalisation qu'on propose ici on considère que les
caractéristique en entrée constitue un espace d'états $\Omega$. On
considère de plus que celui-ci vient avec une fonction $\lambda$ qu'on
désigne par \textit{label} et qui nous indique la caractéristique
qu'on cherche à prédire. On a supposé que le dataset en entrée
contient cette information. Dans ce contexte un classificateur est une
variable aléatoire $X : \Omega \to F$ où $F$ est l'espace des labels
qu'on cherche à attribuer à nos entrées. L'espace $\Omega$ est supposé
fini probabilisé muni d'une probabilité $\mbb{P}$. C'est cette
probabilité qui fait office de \textit{proportion}.

On suppose que $X$ est donnée et on souhaite étudier trois quantités
qui sont liées à l'évaluation de la qualité d'un classificateur.

\paragraph{Cas binaire.} On se limite en un premier temps au cas d'un
classificateur binaire, c'est-à-dire que $F = \{0, 1\}$.
\begin{question}
  On appelle \emph{précision totale} d'un classificateur binaire $X$
  la proportion des individus bien classés, c'est-à-dire ceux où les
  réponses de $X$ et $\lambda$ concident.
  \begin{enumerate}
  \item Exprimer la précision totale de $X$ en s'aidant de la variable
    aléatoire $X - \lambda$.
  \item Déterminer la loi de $X-\lambda$ et interpréter ses deux
    autres valeurs.
  \end{enumerate}
\end{question}
Dans le cas d'un classificateur binaire deux autres quantités
apparaissent relativement naturellement à l'évaluation :
\begin{itemize}
\item la \emph{précision}: proportion des vrais positifs par
  rapport à l'ensemble des bonnes classifications.
\item le \emph{rappel}: proportion des vrais positifs par rapport
  à l'ensemble des positifs (donnés par $\lambda$).
\end{itemize}
\begin{question}
  \begin{enumerate}
  \item Exprmier précision et rappel de $X$.
  \item Étudier l'interconnexion entre précision et rappel ;
    qu'arrive-t-il à l'une si l'autre augmente?
  \end{enumerate}
\end{question}
Dans les problématiques de classification on est toujours attentifs au
fait d'être confrontés à des modèles dont la dépendance au label
$\lambda$ est très faible. C'est pour cette raison qu'on va souvent
chercher à évaluer le \emph{score} d'un classificateur $X$ tels que
$X$ et $\lambda$ définissent des variables aléatoires indépendantes.
\begin{question}
  \begin{enumerate}
  \item Simplifier les expressions de la précision totale, précision
    et rappel dans le cas où $X$ et $\lambda$ sont indépendants.
  \item On suppose que $X$ et $\lambda$ sont indépendants et suivent
    des loi de Bernoulli respectivement de paramètres $p$ et
    $q$. Exprimer les différents scores de $X$ dans ce cas.
  \item Quels resultats numériques obtenez-vous pour chacun des scores
    quand $p=q=0.9$? Qu'en déduisez-vous?
  \end{enumerate}
\end{question}

\paragraph{Cas général.} On revient brièvement vers le cas général. On
suppose désormais que $F$ est l'ensemble discret $\{0, \ldots, k-1\}$.
\begin{question}
  \begin{enumerate}
  \item Exprimer la précision totale de $X$.
  \item Étudiez la loi de $X-\lambda$. Est-elle aussi facilement
    interprétable que dans le cas binaire?
  \item Chercher une variable aléatoire plus adaptée à l'étude des
    problématiques de classifications qui ne sont pas binaires.
  \end{enumerate}
\end{question}


\subsection{Classificateur Bayesien}
\label{sec:Bayesien}

\subsection{\emph{Trade-off} biais variance}
\label{sec:biaisVariance}


% \pretitle{\vspace{-2\baselineskip} \begin{center}}
% \title{%
%   { \huge Solutions des exercices}%
% }
% \posttitle{
% \end{center}
%   \vspace{.5\baselineskip}
%   \rule{\textwidth}{1.5pt}
%   \vspace{-5\baselineskip}
% }

% \maketitle\thispagestyle{fancy}

% \noindent Vous trouverez dans la suite solutions et indications d'une
% partie des exercices de la feuille. Ceci étant majoritairement
% accessibles il vous est suffisant de comparer votre travail au
% résultats que vous retrouverez dans la suite.

% \printsolutions

\end{document}

%%% Local Variables:
%%% mode: latex
%%% TeX-master: t
%%% End:
