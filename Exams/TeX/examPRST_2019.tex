\documentclass[11pt, a4paper]{article}

\usepackage[french]{babel}
\usepackage{fancyhdr}
\usepackage[margin=.8in]{geometry}

\usepackage{Style/TeXingStyle}

\pagestyle{fancy}
\renewcommand{\headrulewidth}{1.5pt}
\renewcommand{\footrulewidth}{0.5pt}
\fancyhead[L]{EPITA\_ING2\_2019\_S8}
\fancyhead[R]{Majeure SCIA}
\fancyhead[C]{PRST}
\fancyfoot[C]{\thepage}
\fancyfoot[L]{Avril 2018}
\fancyfoot[R]{\textbf{Chargé de cours :} \textsc{Bashar~DUDIN}}

\pretitle{\vspace{-.5\baselineskip} \begin{center}}
\title{%
  { \huge Examen de Probabilités}%
}
\posttitle{
\end{center}
  \begin{flushleft}
    \vspace{3\baselineskip} \textit{
      \!\!\emph{Durée de l'épreuve 3h.}\\
      \! \emph{Les documents du cours ne sont pas autorisées, les calculettes non programmables le sont.}  \\
      Le barême est indicatif et peut évoluer de manière
      marginale. Vous avez $40$ points sur l'ensemble du sujet, un
      coefficient adapté sera appliqué à vos résultats. Faites le
      maximum que vous pouvez.}
  \end{flushleft}
  \rule{\textwidth}{1.5pt}
  \vspace{-5\baselineskip}
}
\author{}
\date{}

\pdfinfo{
   /Author (Bashar Dudin)
   /Title  (Examen de probabilités - 2018)
   /Subject (Probabilités)
}

\begin{document}

\maketitle\thispagestyle{fancy}

\section{Cas discret}

Cette première partie se focalise sur l'étude de phénomènes discret.
La première section s'intéresse à la somme de deux lois binômiales. On
se propose dans la seconde d'étudier la loi de probabilité qui régit
la réussite au QCMs de prépa intégrée Epita par une stratégie
aléatoire\footnote{Et hasardeuse!}. Il s'agit de mettre à l'épreuve
quelques raisonnements\footnote{Foireux!} de certains de nos chers
étudiants. On décrit en un premier temnps le contexte dans lequel on
se place.

\subsection{Somme de binômiales}

On se donne deux variables aléatoires indépendantes $X$ et $Y$ suivent
des lois binomiales de paramètres respectifs $(n, p)$ et $(m, p)$ où
$p \in ]0,1[$ et $n$, $m \in \N^*$.

\begin{question}{3}
  Identifier la loi suivie par la variable aléatoire $Z = X + Y$.
\end{question}

\subsection{Du hasard aux QCMs}

Un QCM est composé de $n$ questions numérotées de $1$ à $n$. Pour
chaque question $i$ on a $R_i$ suggestions numérotées de $1$ à
$r_i$. Répondre à une question signifie cocher les suggestions qui
semblent justes. La réponse à la question $i$ est une partie de $R_i$,
éventuellement vide (le cas où l'on choisit de ne pas
répondre). \emph{Il n'y a qu'une seule combinaison de suggestions
  considérée comme la bonne réponse à une question donnée}. Toutes les
autres réponses sont fausses et sinon \emph{NA}\footnote{Non
  Assignée.}, dans le cas où l'étudiant choisi de ne pas répondre. Le
fait de ne pas répondre n'est jamais considéré comme une bonne
réponse.

\begin{flushleft}
  \textit{Pour simplifier l'étude nous allons supposer que toutes les
    questions ont un même nombre de suggestions $r$.}\emph{On
    suppose de plus que les réponses aux questions du QCMs sont
    indépendantes.}
\end{flushleft}

\begin{question}{2}
  Décrire l'espace d'états $\Omega_i$ de l'expérience aléatoire
  \textit{répondre à la question $i$}. En déduire l'espace d'états
  $\Omega$ qui correspond au fait de répondre à toutes les
  questions. \emph{Indiquer le cardinal de chacun de ces espaces
    d'états}.
\end{question}

On désigne par $\rho$ le nombre de réponses (ou non-réponses)
possibles une question\footnote{Toutes les questions ont le même
  nombres de suggestions $r$.}. Autrement dit,
$\rho = \Card{\Omega_i}$. Soit $p \in [0, 1]$. On définit sur chaque
$\Omega$ la probabilité suivante:
\begin{enumerate}
\item[\textbullet]
  $\mbb{P}(\emptyset) = p$
\item[\textbullet]
  $\forall \omega \in \Omega_i\setminus \{\emptyset\}$,
  $\mbb{P}(\omega) = \displaystyle{\frac{1-p}{\rho-1}}$.
\end{enumerate}
La probabilité $p$ est, donc, celle de ne pas répondre.
\begin{question}{2}
  Vérifier que $\mbb{P}$ est bien une probabilité sur
  $\Omega_i$.
\end{question}
Pour chaque $i$ on désigne par $X_i$ la variable aléatoire sur
$\Omega_i$ telle que
\[
X_i(\omega) = \left\{
  \begin{matrix}
    F & \textrm{si $\omega$ est une réponse fausse;}\\
    V & \textrm{si $\omega$ est la bonne réponse;}\\
    NA & \textrm{s'il n'y a pas de réponse.}
  \end{matrix}
  \right.
\]

\begin{question}{4}
  Décrire la loi de $X_i$. Par hypothèse, les variables aléatoires
  $X_i$ sont indépendantes. On note $X$ la variable aléatoire produit
  des $X_i$. Quelle est la probabilité qu'un étudiant donne la bonne
  réponse pour
  \begin{enumerate}
  \item une seulle question;
  \item toutes les questions;
  \item au moins deux questions?
  \end{enumerate}
\end{question}
On s'intéresse désormais à la note de l'étudiant hasardeux. On désigne
par $N_i$ la variable aléatoire sur $\Omega_i$ et à valeurs dans
$\{-1, 0, 2\}$ qui représente la note à la question $i$. Par
définition
\[
N_i(\omega) = \left\{
  \begin{matrix}
    -1 & \textrm{si $\omega$ est une réponse fausse;}\\
    2 & \textrm{si $\omega$ est la bonne réponse;}\\
    0 & \textrm{s'il n'y a pas de réponse.}
  \end{matrix}
  \right.
\]
\begin{question}{1}
  Quel est le lien entre $N_i$ et $X_i$? En déduire que les $N_i$ sont
  des variables aléatoires indépendantes.
\end{question}
La note totale de notre étudiant est $N = \sum_{i=1}^n N_i$. Notre but
est d'étudier l'espérence et la variance de $N$ en passant par l'étude
des séries génératrices des $N_i$. Le point technique réside dans le
fait qu'on a définit une série génératrice dans le cas d'une variable
aléatoire qui prenait ses valeurs dans $\N$, ce qui n'est pas le cas
de $N_i$ (elle peut prendre la valeur $-1$). Pour contourner ce
problème on s'intéresse à la variable aléatoire $K_i = N_i + 1$. Les
$K_i$ prennent leurs valeurs dans $\{0, 1, 3\} \subset \N$. On désigne
par $K$ la variable aléatoire $K = \sum_{i=1}^n K_i$.
\begin{question}{3}
  \begin{enumerate}
  \item Quel est le lien entre $K$ et $N$?
  \item Exprimer l'espérence de $K$ en fonction de celle de $N$.
  \item Faire de même avec la variance.
  \end{enumerate}
\end{question}
On s'intéresse désormais à la série génératrice de $K$.
\begin{question}{7}
  \begin{enumerate}
  \item Donner l'expression de la série génératrice de $K_i$ en
    fonction de $p$ et $\rho$.
  \item En déduire la série génératrice de $K$.
  \item En déduire l'espérence et la variance de $N$ en fonction de
    $p$ et $\rho$.
  \item Décrire la variance et l'espérence pour les cas
    $\rho \in \{2, 3, 4\}$. Que pouvons-nous dire pour des $\rho > 4$?
  \end{enumerate}
\end{question}

\section{Lois continues}

Le travail qui suit porte sur les variables aléatoires à densités.

\subsection{Un calcul de densité}

On considère la variable aléatoire de densité $f(x) = 3x^3\bOne_{]0,1]}(x)$.
\begin{question}{3}
  On désigne par $Y$ la variable aléatoire $Y = 1/X$.
  \begin{enumerate}
  \item Calculer l'espérence de $Y$.
  \item Déterminer la fonction de répartition $F_Y$ de $Y$ et en
    déduire sa densité $f_Y$.
  \end{enumerate}
\end{question}

\subsection{Lois exponontielles}

On se donne deux variables aléatoires indépendantes $X$ et $Y$ de lois
exponentielles respectivement de paramètres $\lambda$ et $\mu$. Pour
rappel, la loi exponentielle de paramètre $\lambda$ est donnée par la
densité
\[
f_X(x) = \lambda e^{-\lambda x}\bOne_{\R_+}.
\]
On s'intéresse aux variables aléatoire $Z = \min\{X, Y\}$ et $S = X+Y$.
\begin{question}{3}
  \begin{enumerate}
  \item Exprimer $\{Z \geq t \}$ en fonction de $\{ X \geq t\}$ et
    $\{Y \geq t\}$.
  \item Montrer que $Z$ suit une loi exponentielle de paramètre
    $\lambda$ + $\mu$.
  \end{enumerate}
\end{question}
On s'intéresse désormais à la variable aléatoire $S$.
\begin{question}{4}
  \begin{enumerate}
  \item Calculer $\mbb{V}(X)$ et $\mbb{E}(X)$. Vérifier, en particulier, que $\mbb{V}(X) = \mbb{E}(X)^2$.
  \item Calculer la variance et l'espérence de $S$.
  \item En déduire que $S$ ne peut pas suivre une loi exponentielle.
  \end{enumerate}
\end{question}


\subsection{Loi uniforme sur un triangle}

On note T l'intérieur du triangle dans $\R^2$ délimité par les points
de coordonnées $(0,0)$, $(1,0)$ $(0,1)$. On note $Z = (X, Y)$ le
couple de variables aléatoires de loi uniforme sur T. Autrement dit
dont la denisté est $2\bOne_T$.

\begin{question}{4}
  \begin{enumerate}
  \item Calculer les lois marginales de $X$ et de $Y$.
  \item Calculer la covariance des variables aléatoires $X$ et
    $Y$. Sont-elles indépendantes?
  \end{enumerate}
\end{question}

\subsection{Loi uniforme sur un carré}

On considère la loi $Z = (X, Y)$ définie comme la loi uniforme sur
$[0, 1]^2$ ; sa fonction de répartion sur $\R^2$ est donnée par la
fonction indicatrice $\bOne_{[0,1]^2}$.
\begin{question}{4}
  Montrer que les variables aléatoires  $X$ et $Y$ sont indépendantes,
  de même loi uniforme sur $[0, 1]$.
\end{question}

\end{document}

%%% Local Variables:
%%% mode: latex
%%% TeX-master: t
%%% End:
