\documentclass[12pt, a4paper]{article}

\usepackage[francais]{babel}
\usepackage{fancyhdr}
\usepackage[margin=.8in]{geometry}

\usepackage{Style/TeXingStyle}

\pagestyle{fancy}
\renewcommand{\headrulewidth}{1.5pt}
\renewcommand{\footrulewidth}{0.5pt}
\fancyhead[L]{EPITA\_ING2\_2020\_S8}
\fancyhead[R]{Majeure SCIA}
\fancyhead[C]{PRSTIA}
\fancyfoot[C]{\thepage}
\fancyfoot[L]{2019}
\fancyfoot[R]{\textbf{Chargé de cours:} \textsc{Bashar~DUDIN}}

\pretitle{\vspace{-2.5\baselineskip} \begin{center}}
\title{%
  {\huge Dénombrement - Fiche Théorique}%
}
\posttitle{
\end{center}
\rule{\textwidth}{1pt}
}
\author{}
\date{}

\pdfinfo{
   /Author (Bashar Dudin)
   /Title  (Dénombrement)
   /Subject (SCIA - Probabilités)
}

\begin{document}

\maketitle\thispagestyle{fancy}

\begin{abstract}
  Cette annexe constitue un rappel succint et un peu sec des contenus
  de cours en dénombrement. C'est un rappel utile des démarches
  théoriques qui sous-tendent les différentes stratégies de comptage
  que l'on aborde dans le cours \texttt{PRSTIA}.
\end{abstract}


En mathématiques le dénombrement est l'acte de \emph{compter} ou
d'\emph{énumérer} les éléments d'un ensemble. On compte le nombre de
produits restant dans un rayon de supermarché, le nombre d'étudiants
ayant réussi un examen, le nombre d'espèces animales
disparues\footnote{Exemple qui n'est en rien lié au précédent :).}
... etc.

Ces exemples, intuitifs, ne sont pas la limite de ce qu'on
souhaiterait pouvoir \emph{énumérer}. Dénombrer des ensembles finis et
une manière de quantifier leurs <<~grosseurs~>>. De ce point de
vue, il est naturel de se poser la question de savoir si l'ensemble
des entiers relatifs $\Z$ est plus <<~gros~>> que celui des
entiers relatifs $\N$. Ou encore si l'ensemble des nombres réels $\R$
et plus grand que celui des rationnels $\Q$. C'est un aspect qu'on ne
verra qu'en surface pour commencer. Son importance pour la suite de
vos études va, cela dit, nous forcer à le prendre en compte dans les
définitions qu'on donnera.

\vspace{\baselineskip}
Pour répondre aux questions laissées en suspens dans ce propos
introductif, on verra dans la suite que $\Z$, $\N$ et $\Q$ sont aussi
<<~gros~>> les uns que les autres mais que $\R$ est plus <<~gros~>>.

\section{Ensembles finis}

Dans le cas des ensembles finis la notion de cardinal correspond à
celle de nombre d'éléments appartenant à l'ensemble. Il suffira de
garder cette idée en tête en un premier temps, on reviendra sur la
subtilité de cette définition par la suite. Le but de cette section
est d'aborder la question de \emph{comment formaliser le fait de
  compter?}.

\begin{defn}
  Un ensemble $E \neq \emptyset$ est fini s'il existe une bijection de
  $E$ dans $\{1, 2, \ldots, n\}$ pour un entier $n > 0$. Par
  convention l'ensemble vide $\emptyset$ est également fini.
\end{defn}
\noindent L'entier $n$ de la définition précédente n'est pas encore
garanti d'être unique ; du moins la définition ne le dit pas.
\begin{lem}
  \label{lem:cardinj}
  Soient $(n, m)$ un couple d'entiers naturels non nuls. S'il existe
  une injection de $\{1, \ldots n\}$ dans $\{1, \ldots, m\}$ alors
  $n \leq m$.
\end{lem}
\begin{proof}
  On raisonne par récurrence sur $m$. Dans le cas où $m = 1$, toute
  application de $\{1, \ldots, n\}$ sur $\{1\}$ est constante. Une
  telle application est injective si et seulement si $n = 1$.

  Dans le cas général, soit $f$ une application injective de
  $\{1, \ldots, n\}$ vers $\{1, \ldots, m\}$. Quitte à renuméroter les
  éléments de $\{1, \ldots, m\}$ on peut supposer que $f(n) =
  m$. Comme $n$ est dans ce cas l'unique antécédant de $m$ par $f$, la
  restriction de $f$ à $\{1, \ldots, n-1\}$ a une image dans
  $\{1, \ldots, m-1\}$. Par hypothèse de récurrence il en vient que
  $n-1 \leq m-1$, d'où $n \leq m$.
\end{proof}
\begin{cor}
  \label{c:bonnedefcard}
  Étant donné un ensemble fini $E \neq \emptyset$, l'entier naturel
  $n$ tel que $E$ est en bijection avec $\{1, \ldots, n\}$ est unique
  ; c'est le cardinal de $E$. Par convention le cardinal de
  $\emptyset$ est $0$.
\end{cor}
\begin{proof}
  Il n'y a rien à montrer pour le cas de l'ensemble vide car la
  question est réglée par convention. On cherche à montrer que s'il
  existe des bijections $f$ et $g$ respectivement de $E$ dans
  $\{1, \ldots, n\}$ et $\{1, \ldots, m\}$, pour deux entiers naturels
  $n$ et $m$, alors $n = m$. Sous de telles hypothèses on obtient les
  deux bijections (et donc injections) $f^{-1}\circ g$ et
  $g^{-1}\circ f$ respectivement de $\{1, \ldots, n\}$ dans
  $\{1, \ldots, m\}$ et inversement. D'après le lemme \ref{lem:cardinj}
  $n \geq m$ et $m \geq n$, donc $m = n$.
\end{proof}

\begin{nota}
  Le cardinal d'un ensemble $E$ sera noté $\Card{E}$ ou encore
  $\htag{E}$.
\end{nota}

La pratique du dénombrement consiste à calculer les cardinaux
d'ensembles d'intérêt. Cette pratique se base sur un nombre de
principes élémentaires qu'on déclinent suivant la stratégie de travail
qu'on adopte. On reviendra sur ces principes, souvent intuitifs, à la
section \ref{sec:principesEns}. On se limite pour l'instant à des
exemples qu'on peut écrire explicitement.

\begin{question}
  Déterminer les cardinaux des ensembles décrits par la suite:
  \begin{itemize}
  \item Le nombre d'entiers pairs plus petits que $12$. Ceux plus
    petits que $13$? Comment procéder dans le cas général?
  \item Le nombre de points dont les deux coordonées sont entières
    comprises entre $1$ et $7$ et qui se trouvent sur la droite
    $ 2y = 1 - x$.
  \item Le nombre de mots (sans qu'ils aient nécessairement du sens)
    qu'on peut écrire en réordonnant les lettres de <<~abba~>>.
  \end{itemize}
\end{question}

\begin{hyp}
  À moins de faire explicitement mention de contraire les ensembles
  qu'on manipulent sont \emph{finis}.
\end{hyp}

\section{Principes ensemblistes}

\label{sec:principesEns}
Certains principes de compatibilités des opérations ensemblistes à la
notion de cardinal sous-tendent l'essentiel des raisonnements qu'on
aborde dans ce cours. Ces principes sont intuitifs, même si les
preuves de ceux-ci restent assez formelles.\footnote{Il est possible
  de se contenter de dessin pour justifier chacun des principes en
  question. Garder en extension les preuves formelles de celles-ci.}

\begin{prop}
  Soit $E$ un ensemble de cardinal fini et $\{A_i\}_{i \in I}$ une
  partition finie ($I$ de cardinal fini) de $E$. Alors
  \begin{equation}
    \label{eq:cardpartition}
    \Card{E} = \sum_{i \in I} \Card{A_i}.
  \end{equation}
\end{prop}
\begin{proof}
  On se contente d'une esquisse de la preuve. Une preuve par
  récurrence serait plus adaptée si l'on souhaite en donner une
  particulièrement rigoureuse, mais il ne semble pas que celle-ci
  apporte une visibilité plus accrue quant à la justification de cette
  proposition.

  Si $I$ est vide $E$ est de même et la relation
  \ref{eq:cardpartition} est satisfaite. Supposons que $I$ n'est donc
  pas vide. Par hypothèse sur $I$ on peut identifier $I$ à
  $\{1, \ldots, p\}$ pour $p \geq 1$. De même chaque $A_i$ est en
  bijection avec $\{1, \ldots, n_i\}$ pour $n_i > 0$. On suppose ici
  les $A_i$ non-vide, ce qui ne change rien au résultat. On construit
  pour chaque $i \in \{1, \ldots, p\}$ une bijection obtenue par
  \emph{shift} de $\{1, \ldots, n_i\}$ vers
  $\{s_i + 1, \ldots, s_i + n_i\}$ où $s_i = \sum_{k=1}^{i-1} n_k$
  (par convention ici $s_1 = 0$). En regroupant ces bijections, on
  obtient une bijection de $E$ sur $\{1, \ldots, s_p\}$. Or $s_p$ est
  précisément le membre de droite de \ref{eq:cardpartition}. Pour une
  description plus imagée de cette proposition voir
  figure \ref{fig:partition}.
\end{proof}
\begin{figure}
  \centering
  \colorlet{grayedarea}{gray!20}
  \begin{tikzpicture}
    \fill[grayedarea] (1, 0) rectangle (2, 4) ;
    \draw (2, 0) grid[step=1] (3, 8);
    \draw (0, 0) grid[step=1] (1, 7);
    \draw (1, 0) grid[step=1] (2, 4);
    \draw (3, 0) grid[step=1] (4, 6);
    \draw (4, 0) grid[step=1] (5, 7);
    \draw (5, 0) grid[step=1] (6, 5);
    \node at (1.5, -.5) {$i = 1$};
    \node at (2.5, 8.5) {$A_2$};
    \node at (.5, 7.5) {$A_0$};
    \node at (1.5, 4.5) {$A_1$};
    \node at (3.5, 6.5) {$A_3$};
    \node at (4.5, 7.5) {$A_4$};
    \node at (5.5, 5.5) {$A_5$};
  \end{tikzpicture}
  \caption{Cardinal d'un ensemble fini via une partition.}
  \label{fig:partition}
\end{figure}
\begin{cor}
  Soit $F$ un sous-ensemble d'un ensemble fini $E$. Alors
  \[
    \Card{F} \leq \Card{E}.
  \]
\end{cor}
\begin{proof}
  La décomposition
  \[
    E = F \sqcup \big( E \setminus F \big),
  \]
  qu'on retrouve représentée figure \ref{fig:subset}, donne
  \[
    \Card{E} = \Card{F} + \underbrace{\Card{E \setminus F}}_{\geq 0}.
  \]
  D'où la relation recherchée.
\end{proof}
\begin{figure}[h]
  \centering
  \colorlet{circle edge}{gray}
  \colorlet{circle area}{gray!20}

  \tikzset{filled/.style={fill=circle area, draw=circle edge, thick},
    outline/.style={draw=circle edge, thick}}

  \setlength{\parskip}{5mm}

  \begin{tikzpicture}[scale=0.9]
    % Set E or F
    \draw[filled] (0, 0) circle (2cm)
                  (-.5, -.5) circle (1cm) node {$F$};
    \node at (0.5, 1) {$E$};
    % Set E \ F
    \begin{scope}
      \draw[filled] (5.5, -.5) circle (1cm) node {$F$};
    \end{scope}
    \draw[outline] (5.5, -.5) circle (1cm)
                   (6, 0) circle (2cm);
    \node at (6.5, 1) {$E$};
    % Set F
    \begin{scope}
      \clip (12, 0) circle (2cm);
      \draw[filled, even odd rule] (12, 0) circle (2cm)
                                   (11.5, -.5) circle (1cm) node {$F$};
    \end{scope}
    \draw[outline] (12, 0) circle (2cm)
                   (11.5, -.5) circle (1cm);
    \node at (12.5, 1) {$E$};
    \node at (3, 0) {$=$};
    \node at (9, 0) {$\setminus$};
  \end{tikzpicture}
  \caption{Cardinal d'ensembles finis et inclusion.}
  \label{fig:subset}
\end{figure}
\begin{cor}
  Soient $F$ et $G$ deux sous-ensembles d'un ensemble fini $E$. Alors
  \[
    \Card{F \cup G} = \Card{F} + \Card{G} - \Card{F\cap G}.
  \]
\end{cor}
\begin{proof}
  On peut partitionner l'union $F \cup G$ de la manière suivante (voir
  figure \ref{fig:union}):
  \[
    F\cup G = F\sqcup \Big( G\setminus (F\cap G) \Big).
  \]
  D'après le premier principe on obtient
  \begin{equation}
    \label{eq:carddisjoint}
    \Card{F\cup G} = \Card{F} + \Card{G\setminus (F\cap G)}.
  \end{equation}
  Or
  \[
    G = \Big(G\setminus (F\cap G)\Big) \sqcup (F\cap G).
  \]
  Donc
  \[
    \Card{G} = \Card{G\setminus (F\cap G)} + \Card{(F\cap G)}.
  \]
  On retrouve la relation rechercée en reinjectant cette relation dans
  l'équation \ref{eq:carddisjoint}.
\end{proof}
\begin{figure}[h]
  \centering
  \colorlet{circle edge}{gray}
  \colorlet{circle area}{gray!20}

  \tikzset{filled/.style={fill=circle area, draw=circle edge, thick},
    outline/.style={draw=circle edge, thick}}

  \setlength{\parskip}{5mm}

  \begin{tikzpicture}[scale=0.9]
    % Set E or F
    \draw[filled] (0, 0) circle (1.5cm) node {$E$}
                  (0:2cm) circle (1.5cm) node {$F$};
    \node[anchor=south] at (1, 1.5) {$E \cup F$};
    % Set E
    \begin{scope}
      \draw[filled] (6, 0) circle (1.5cm) node {$E$};
    \end{scope}
    \draw[outline] (6, 0) circle (1.5cm)
                   (8, 0) circle (1.5cm) node {$F$};
    \node[anchor=south] at (7, 1.5) {$E$};
    % Set E but not F
    \begin{scope}
      \clip (14, 0) circle (1.5cm);
      \draw[filled, even odd rule] (12, 0) circle (1.5cm)
                                   (14, 0) circle (1.5cm) node {$F$};
    \end{scope}
    \draw[outline] (12, 0) circle (1.5cm) node {$E$}
                   (14, 0) circle (1.5cm);
    \node[anchor=south] at (13, 1.5) {$E\setminus F$};
    \node at (4, 0) {$=$};
    \node at (10, 0) {$\sqcup$};
    \node at (4, -3) {Le dernier terme de cette décomposition s'écrit également};
    % Set E but not F
    \begin{scope}
      \clip (2, -6) circle (1.5cm);
      \draw[filled, even odd rule] (0, -6) circle (1.5cm)
                                   (2, -6) circle (1.5cm) node {$F$};
    \end{scope}
    \draw[outline] (0, -6) circle (1.5cm) node {$E$}
                   (2, -6) circle (1.5cm);
    \node[anchor=south] at (1, -4.5) {$E\setminus F$};
    \begin{scope}
      \draw[filled] (8, -6) circle (1.5cm) node {$F$};
    \end{scope}
    \draw[outline] (8, -6) circle (1.5cm)
                   (6, -6) circle (1.5cm) node {$E$};
    \node[anchor=south] at (7, -4.5) {$F$};
    \begin{scope}
        \clip (12, -6) circle (1.5cm);
        \fill[filled] (14, -6) circle (1.5cm);
    \end{scope}
    \draw[outline] (12, -6) circle (1.5cm) node {$E$};
    \draw[outline] (14, -6) circle (1.5cm) node {$F$};
    \node[anchor=south] at (13, -4.5) {$E \cap F$};
    \node at (4, -6) {$=$};
    \node at (10, -6) {$\setminus$};
  \end{tikzpicture}
  \caption{Cardinal de l'union de deux ensembles finis.}
  \label{fig:union}
\end{figure}
\begin{prop}
  Soient $\{A_i\}_{i \in I}$ une famille finie ($I$ est de cardinale
  fini) d'ensembles finis. On note $P$ le produit cartésien des $A_i$
  pour $i \in I$. Alors
  \[
    \Card{P} = \prod_{i \in I} \Card{A_i}.
  \]
\end{prop}
\begin{proof}
  On raisonne par récurrence sur le cardinalt de $I$. Par convention,
  le produit d'une famille indexée par $\emptyset$ est $0$. La
  relation est donc vérifiée dans ce cas. Ce cas étant une situation
  marginale, et un peu délicate à entendre, on se contente de débuter
  notre raisonnement au cas où $I$ est de cardinal $1$. Il n'y a dans
  ce cas rien à montrer ; les deux ensembles impliqués dans les
  membres de gauche et droite de l'égalité sont les mêmes. On se donne
  désormais un ensemble $I$ de cardinal $n+1$ et $i_0 \in I$. On a
  \[
    P = \bigsqcup_{k \in A_{i_0}} {k}\times\left(\prod_{i \in
        I\setminus i_0} A_i\right).
  \]
  D'après notre premier principe cela implique
  \[
    \Card{P} = \Card{A_{i_0}} \times \Card{\prod_{i \in I\setminus i_0} A_i}.
  \]
  On conclut par hypothèse de récurrence au rang $n$. La démarche
  qu'on vient de mettre en évidence est représentée figure
  \ref{fig:product}.
\end{proof}
\begin{figure}
  \centering
  \colorlet{grayedarea}{gray!20}
  \begin{tikzpicture}
    \fill[grayedarea] (.5, -.5) rectangle (1.5, 8.5) ;
    \draw[dashed] (0, 0) grid[step=1] (7, 8);
    \foreach \x in {0, 1, ..., 7}{
      \foreach \y in {0, 1, ..., 8}{
        \node[draw, circle, inner sep=2pt, fill] at (\x, \y) {};
      }
    }
    \node at (1, -1) {$i_0$};
    \node at (-1, 4) {$A_2$};
    \node at (3.5, -1) {$A_1$};
  \end{tikzpicture}
  \caption{Cardinal d'un produit cartésien d'ensembles finis.}
  \label{fig:product}
\end{figure}
\section{Les dénombrements de référence}

Au côtés des principes étudiés précédemment il existe quelques
dénombrements de référence qu'il est avisé de connaître. Ils sont
d'utilisation constante dans la pratique.

\subsection{Nombres de bijections}

Comme le titre le dit : on souhaite calculer le nombre de bijections
entre deux ensembles finis donnés. D'après \ref{c:bonnedefcard}
l'ensemble des bijections entre deux ensembles finis est non vide si
et seulement si ils ont même cardinaux.
\begin{question}
  \label{q:bijection}
  Dénombrer l'ensemble des bijections de l'ensemble des $3$ points
  cardinaux $\mc{E} = \{E, S, W\}$ vers celui des couleurs
  $\mc{F} = \{\spadesuit, \heartsuit, \clubsuit\}$ d'un jeux de
  cartes.
\end{question}
\begin{rem}
  Le comptage précédent revient à compter les bijections de
  $\{1, 2, 3\}$ dans lui même. En effet, il existe des bijections
  $f$ et $g$ respectivement de $\mc{E}$ et $\mc{F}$ vers
  $\{1, 2, 3\}$. À toute bijection $\beta : \mc{E} \to \mc{F}$
  correspond la bijection $g\circ\beta\circ f^{-1}$ de $\{1, 2, 3\}$
  sur lui-même. Inversement une bijection $\gamma$ de $\{1, 2, 3\}$
  dans lui même donne la bijection $g^{-1}\circ\gamma\circ f$ de
  $\mc{E}$ dans $\mc{F}$.
  \[
    \begin{tikzcd}
      \{1, 2, 3\}
      \arrow[rrr, bend right, "g\circ\beta\circ f^{-1}" below] &
      \mc{E} \arrow[l, "f" above] \arrow[r, "\beta"] &
      \mc{F} \arrow[r, "g"] &
      \{1, 2, 3\}
    \end{tikzcd}
    \qquad
    \begin{tikzcd}
      \mc{E} \arrow[r, "f"]
      \arrow[rrr, bend right, "g^{-1}\circ\gamma\circ f" below] &
      \{1, 2, 3\} \arrow[r, "\gamma"] &
      \{1, 2, 3\} &
      \mc{F} \arrow[l, "g" above]
    \end{tikzcd}
  \]
\end{rem}
Le raisonnement précédent est général. Étudier les bijections entre
deux ensembles finis (nécessairement de même cardinal, si l'on espère
au moins une bijection) revient à étudier les bijections de
$\{1, \ldots n\}$ dans lui même.
\begin{nota}
  Soit $n \in N^*$. On appelle \emph{permutation} de
  $\{1, \ldots, n\}$ toute bijection de $\{1, \ldots, n\}$. L'ensemble
  des permutations de $\{1, \ldots, n\}$ est noté $S_n$ ou encore
  $\mf{S}_n$.
\end{nota}
Cherchons à se donner une permutation $\sigma$ de $\{1, \ldots, n\}$
pour un entier $n > 1$. En supposant qu'on connaisse l'image de $1$
par $\sigma$, il nous reste plus qu'à trouver une bijection de
$\{2, \ldots, n\}$ dans $\{1, \ldots, n\}\setminus \sigma(1)$. Au
réordonnement prêt des éléments de ces deux ensembles on vient de se
ramener à trouver une permutation de $\{1, \ldots, n-1\}$. On peut
donc écrire
\[
  \mf{S}_n = \bigsqcup_{i=1}^n \{i\}\times \mf{S}_{n-1}.
\]
où la première coordonnées correspond à l'image de $1$. On en déduit
donc, pour tout entier $n > 1$
\[
  \Card{\mf{S}_n} = n \times \Card{\mf{S}_{n-1}}.
\]
Un calcul du cardinal de $\mf{S}_1$ permet de calculer le nombre de
permutations de $\{1, \ldots, n\}$ par récurrence.
\begin{defn}
  Soit $n \in \N$. On appelle \emph{factoriel} $n$, noté $n!$
  l'entier naturel défini récursivement par
  \begin{itemize}
  \item[\textbullet] $0! = 1$ ;
  \item[\textbullet] $n! = n \times (n-1)!$.
  \end{itemize}
\end{defn}
\begin{prop}
  Soit $n \in \N^*$. L'ensemble des permutations $\mf{S}_n$ est de
  cardinal $n!$.
\end{prop}
\begin{question}
  Calculer le nombre
  \begin{itemize}
  \item de manières d'ordonner les listes $[1; 2; 3; 3; 5; 6]$,
    $[1; 2; 3; 3; 5; 6]$ ;
  \item de mots qu'on crée à partir du mot <<~saperlipopette~>>.
  \end{itemize}
\end{question}

\subsection{Arrangements}

Le nombre d'arrangements est le nombre d'\emph{injections} de
$\{1, \ldots, k\}$ dans $\{1, \ldots, n\}$. Comme dans le cas des
permutations ; cela sert de modèle pour le nombre d'injections entre
deux ensembles finis quelconques. Toujours d'après
\ref{c:bonnedefcard} pour que l'ensemble des arrangements soit non
vide il faut $k \leq n$.

Soient $k$, $n$ deux entiers tels que $n > 1$ et $k < n$. On note
$A_{k, n}$ l'ensemble des arrangements de $\{1, \ldots, k\}$ dans
$\{1, \ldots, n\}$. Par un raisonnement identique au cas des
permutations on a la décomposition
\[
  A_{k, n} = \bigsqcup_{i=1}^n \{i\}\times A_{k - 1, n-1}.
\]
On en déduit donc
\[
  \Card{A_{k, n}} = n\times \Card{A_{k - 1, n-1}}.
\]
L'ensemble $A_{1, n - k + 1}$ correspond aux injections de $\{1\}$
dans $\{1, \ldots, n - k + 1\}$.
\begin{prop}
  Étant donné deux entiers positifs $k$, $n$ tels que $k \leq n$ alors
  \[
    \Card{A_{k, n}} = n\times (n-1) \times \cdots \times (n - k + 1) = \frac{n!}{(n-k)!}.
  \]
\end{prop}
\begin{question}
  Exprimer le nombre de sous-chaînes de longueur $8$ et sans
  répétitions qu'on peut extraire de l'alphabet.
\end{question}

\subsection{Combinaisons}

Quand on étudie le problème de dénombrer les sous-chaînes de longueur
$8$ de l'alphabet, on est préoccupé par l'ordre des lettres qui
apparaissent dans celles-ci. Il n'en serait pas de même si l'on
s'intéresse au nombres de manière de composer un sac de courses
contenant $8$ produits à partir d'un ensemble fixé de produits. Qu'on
prenne les tomates en premier ou en second ne change rien à la
composition du sac. Dans le premier cas on s'intéresse au $8$-uplets de
lettres de l'alphabet, dans le second on s'intéresse aux
sous-ensembles ayant $8$ éléments composés de produits à disposition.

Pour tout $k$, $n$ des entiers naturels tels que $k \leq n$ on note
$C_{k, n}$ l'ensemble des parties ayant $k$ éléments de
$\{1, \ldots, n\}$. On a une application
$\phi : A_{k, n} \to C_{n, k}$ qui envoie un $k$-uplet de $A_{k, n}$
sur la partie de cardinal $k$ correspondante dans $\{1, \ldots,
n\}$. L'image réciproque d'un quelconque élément de $C_{k. n}$
correspond à l'ensemble des manière d'ordonner les $k$ éléments de la
partie concernée. Elle est de cardinal $k!$, on a donc
\[
  A_{k, n} = k!\times C_{k, n}.
\]
\begin{nota}
  Soient $k$, $n$ deux entiers naturels avec $k \leq n$. On note
  $\binom{n}{k}$ la quantité $\frac{n!}{k!(n-k)!}$. Cette quantité est
  qualifiée de \emph{coefficient binômial}\footnote{La raison pour
    cette dénomination apparaîtra de manière plus claire par la
    suite.}.
\end{nota}

\begin{prop}
  Étant donné des entiers naturels $k$, $n$ tels que $k \leq n$ on a
  \[
    \Card{C_{k, n}} = \frac{n!}{k!(n-k)!} = \binom{n}{k}.
  \]
\end{prop}
\begin{question}
  Calculer le nombre
  \begin{itemize}
  \item d'arêtes qu'on peut avoir dans un graphe ayant $3$ sommets ;
  \item de fa\c{c}on d'avoir deux entiers naturels plus petits que
    $10$ qui somment à $12$.
  \end{itemize}
\end{question}

Les coefficients binômiaux satisfonts des propriétés algébriques qui
en facilitent parfois la manipulation et l'usage.
\begin{prop}
  \label{prop:binom}
  Pour tout entière naturel $n \in \N$ on a:
  \begin{itemize}
  \item[\textbullet]
    $\forall k \in \N$, $k \leq \N$,
    \[
      \binom{n}{k} = \binom{n}{n-k}.
    \]
  \item[\textbullet]
    Si $n > 0$, $\forall k \in \N$, $k \leq \N$,
    \[
      \binom{n}{k} = \binom{n-1}{k-1} + \binom{n-1}{k}.
    \]
  \end{itemize}
\end{prop}
\noindent La première propriété est évidente, la seconde se constate
en effectuant explicitement la somme des termes à droite de
l'égalité. Elle est souvent appelée relation de Pascal\footnote{Vous
  êtes invités à vous aider de Wikipedia pour rechercher ce qu'est le
  triangle de Pascal.}.
\begin{rem}
  Les propriétés proposition \ref{prop:binom} permettent de réduire
  les temps de calculs ayant pour objectif de générer les coefficients
  binômiaux:
  \begin{itemize}
  \item la première propriété montre qu'il est suffisant de calculer ses
    coefficients pour $k \in \{0, \ldots, \lceil n / 2 \rceil\}$ ;
  \item la seconde permet d'accélerer les calculs des coefficients
    binômiaux en stockant les valeurs calculées aux étapes
    précédantes. Il faut pour cela remarquer que $\binom{n}{k}$
    apparaît dans les calculs de $\binom{n+1}{k+1}$ et
    $\binom{n+1}{k}$. C'est un processus similaire à l'implémentation
    en complexité linéaire des termes de la suite de Fibonacci.
  \end{itemize}
\end{rem}

\subsection{Formule du binôme de Newton}

La formule du binôme de Newton est l'une des premières manifestations
des coefficients binômiaux dans un contexte algébrique. Cette
apparation permet dans des contextes plus généraux (comme les séries
génératrices en probabilités) de manipuler des quantités
combinatoires, issues de problèmes de dénombrement, algébriquement.
\begin{prop}
  Soit $n$ un entier naturel non nul. Pour tout $x$, $y \in \R$ on a
  \[
    (x + y)^n = \sum_{k=0}^n\binom{n}{k} x^k y^{n-k}.
  \]
\end{prop}
On peut montrer cette formule par récurrence en s'aidant de la
relation de Pascal pour l'hérédité. Voici une heuristique plus à même
de la justifier : tous les termes de la somme contiennent $n$ monômes,
chacun venant d'un facteur différent de la puissance. Le terme
$x^ky^{n-k}$ apparaît autant de fois qu'on a de manière de choisir
$k$-fois $x$ dans les $n$ facteurs de la puissance. C'est donc en
bijection avec le nombre de sous-ensembles à $k$ élément d'un ensemble
à $n$ éléments.

\begin{exmp}
  \label{exmp:binom}
  Le nombre des parties d'un ensemble de cardinal $n$ est $2^n$. En
  effet, on cherche à calculer la somme des coefficients binômiaux
  $\binom{n}{k}$ pour $k \in \{0, \ldots, n\}$. Ainsi
  \[
    \sum_{k=0}^n \binom{n}{k} = \sum_{k=0}^n \binom{n}{k}1^k1^{n-k} =
    (1+1)^n = 2^n
  \]
\end{exmp}
\begin{exmp}
  Le nombre des parties de cardinal pair d'un ensemble de cardinal
  $n > 0$ est $2^{n-1}$. On remarque en un premier temps que
  \[
    0 = (1 - 1)^n = \sum_{k=0}^n \binom{n}{k}(-1)^k.
  \]
  D'où
  \[
    \sum_{0 \leq 2p \leq n} \binom{n}{2p} =  \sum_{0 \leq 2p+1 \leq n} \binom{n}{2p+1}.
  \]
  On trouve le résultat rechercé en réinjectant dans la relation
  obtenue à l'exemple \ref{exmp:binom}.
\end{exmp}

\section{Extension au cas infini}

L'ensemble des entiers naturels $\N$ n'est pas fini. En effet si tel
était le cas, il existerait une bijection $f$ de $\{1, \ldots, n\}$
sur $\N$ pour un entier $n > 0$. L'image de $f$ étant finie on peut
écrire un programme qui nous en calcul le maximum. Le successeur du
maximum n'est pas dans l'image, $f$ n'est donc pas surjective.

On aimerait cela dit être en mesure de comparer l'ensemble des entiers
naturels à celui des entier relatifs, rationnels ou réels. On va
devoir, dans ce but, remodeler la définition de cardinal.

\subsection{La notion de cardinal en général}

\begin{defn}
  Deux ensembles $F$ et $E$ sont dits être de même cardinal s'il
  existe une bijection de l'un sur l'autre. On désinge par $\mc{R}$
  cette relation.
\end{defn}
\noindent On désignge par $\mc{U}$ une classe (propre)\footnote{Pour
  nous, cela n'est qu'une collection d'ensembles. Cette notion est
  cependant nécessaire pour éviter le paradoxe de Russel. Wikipedia
  est votre ami pour un peu plus de culture.} contenant les ensembles
mathématiques d'usage courant $\N$, $\Q$, $\R$, $\C$ leurs unions,
intersections, sous-ensembles et produits.
\begin{prop}
  La relation $\mc{R}$ est une relation d'équivalence sur $\mc{U}$.
\end{prop}
\begin{defn}
  Le cardinal d'un ensemble dans $\mc{U}$ est sa classe d'équivalence
  suivant la relation $\mc{R}$.
\end{defn}
\begin{rem}
  Dans le cas fini, un ensemble $E$ a un unique représentant de la
  forme $\{1, \ldots, n\}$ dans sa classe d'equivalence suivant
  $\mc{R}$. On identifie dans ce cas le cardinal de $E$ à ce
  $n$. C'est de la sorte qu'on retrouve la définition donnée au cours
  de la section précédente.
\end{rem}

\subsection{Les ensembles dénombrables}

\begin{defn}
  Un ensemble est dit dénombrable s'il peut être mis en bijection avec
  une partie de $\N$. Il est non dénombrable sinon.
\end{defn}
\begin{exmp}
  L'ensemble $\N$ est dénombrable par définition.
\end{exmp}
\begin{exmp}
  Les ensembles des entiers pairs, impairs ou des multiples d'un entier
  positifs fixés sont dénombrables.
\end{exmp}
\begin{exmp}
  Tout ensemble de la forme $\{ n \in \N \mid n > M\}$ pour un réel $M$
  est dénombrable.
\end{exmp}
\begin{exmp}
  L'ensemble des entiers relatifs $\Z$ est dénombrable. Il suffit pour
  le voir de construire la bijection de $\N$ sur $\Z$ qui envoie
  bijectivement les entiers pairs sur $\N \subset \Z$ et le entiers
  impairs sur la partie strictement négative. Plus précisément on
  définit la bijection $\psi : \N \to \Z$ par
  \[
    \psi(n) = \left\{
      \begin{matrix}
        p & \textrm{si $n = 2p$} \\
        -(p+1) & \textrm{si $n = 2p+1$}
      \end{matrix}.
    \right.
  \]
\end{exmp}
\begin{exmp}
  Le produit cartésien $\N^2$ est dénombrable. On va se contenter de
  décrire le procédé d'énumération des éléments de $\N^2$. Étant donné
  un entier $k \in \N$, la droite $x + y = k$ contient exactement
  $k+1$ entiers ; $(0, k), (1, k-1), \ldots, (k, 0)$. On compte les
  couples d'entiers sur chacune de ses droites par ordre croissant en
  commen\c{c}ant par $k = 0$. Arrivé à la droite de coefficient
  constant $k$ on a épuisé
  \[
    \sum_{p=0}^k (p+1) = \frac{(k+1)(k+2)}{2}
  \]
  entiers. On reprend le décompte au successeur de cette quantité. Cette
  démarche permet de construire une bijection de $\N$ sur $\N^2$. Elle
  est représentée figure \ref{fig:N2denomb}.
\end{exmp}
\begin{figure}
  \centering
  \colorlet{grayedarea}{gray!20}
  \begin{tikzpicture}
    \draw[dotted, gray] (0, 0) grid[step=1] (7, 7);
    \foreach \x in {0, 1, ..., 7}{
      \foreach \y in {0, 1, ..., 7}{
        \node[draw, circle, inner sep=2pt, fill, gray] at (\x, \y) {};
      }
    }
    \foreach \x in {1, 2, ..., 7}{
      \node at (\x, -.5) {$\x$};
      \node at (-.5, \x) {$\x$};
    }
    \foreach \x in {0, 1, ..., 7}{
      \draw[thick, -latex] (-.5, \x + .5) -- (\x + .5, -.5);
    }
  \end{tikzpicture}
  \caption{Procédé d'énimération de $\N^2$.}
  \label{fig:N2denomb}
\end{figure}
\begin{exmp}
  L'ensemble des nombres rationnels $\Q$ est dénombrable. En effet,
  chaque nombre rationnel $\frac{p}{q}$ a une unique forme réduite
  (forme où $p$ et $q$ n'ont pas de facteurs communs différents de $1$
  ou $-1$). L'application de $\Q$ dans $\Z^2$ qui envoie un nombre
  rationnel vers un le couple composé du numérateur et dénominateur de
  la forme réduite est une bijection sur une partie de $\Z^2$. Donc,
  d'après les exemples précédents, sur une partie de $\N^2$ et donc de
  $\N$.
\end{exmp}

\subsection{Non-dénombrabilité de $\R$}

C'est un point qu'il nous est pas encore possible d'aborder
proprement. Le fait que $\R$ soit d'un cardinal plus \emph{gros} que
$\N$ nécessite plus d'aisance avec les représentations diadiques des
nombres réels et les suites et séries numériques. Ce point, tout comme
le précédent, pourra être revu avec un peu plus de détail lors des
années prochaines. L'un des arguments pour montrer cette
non-dénombrabilité se fait en deux temps. Vous connaissez le
développement décimal d'un nombre réel, chaque chiffre de ce
développement est un chiffre entre $0$ et $9$ compris. Le
développement diadique d'un réel permet d'écrire un nombre réel
uniquement avec des $0$ et des $1$, de la même manière. C'est une
extension de l'écriture en base $2$ des entiers naturels. Cette
écriture d'un nombre réel est unique à un cas prêt à exclure ; le cas
où on n'a que des $1$ dans l'écriture diadique. Si on ne regarde que
les indices des coefficients du développement égaux à $1$ on décrit une
unique partie de $\N$. Cette démarche construit une bijection de $\N$
dans $\mc{P}(\N)$. Vous avez cependant vus que $\N$ et $\mc{P}(\N)$ ne
sont pas en bijection. D'où notre affirmation de départ.


\subsection{Pathologies dans le cas infini}

Les propriétés ensemblistes évoquées dans le cas des ensembles finis
ne s'étendent pas au cadre général. Dans leurs énoncés la majeure
partie introduiraient des sommes ou des produits infinis, la dernière
comparerait des quantités qui ne sont pas réelles et qu'on ne sait
donc pas comparer. Même si dans ce dernier point on gardait une notion
vague de \emph{grosseur} d'un ensemble, on voit bien que l'inclusion
stricte, par exemple $\N \subsetneq \Q$ n'implique pas une différence
sur les cardinaux associés.

\end{document}

%%% Local Variables:
%%% mode: latex
%%% TeX-master: t
%%% End:
